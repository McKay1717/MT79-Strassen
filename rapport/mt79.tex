\documentclass[a4paper,10pt,margin=2in]{report}
\usepackage[utf8]{inputenc}
\usepackage{amsmath}

% Title Page
\title{Multiplication des matrices: Algorithme de Strassen}
\author{Nicolas Iung, Aurelien Blais}

\begin{document}
\maketitle

\chapter{Algorithme de Strassen}
\paragraph{Question 1}
8 multiplications :
\begin{equation}
 c_{11} = a_{11}*b_{11}+a_{12}*b_{21}
\end{equation}
\begin{equation}
 c_{12} = a_{11}*b_{11}+a_{12}*b_{22}
\end{equation}
\begin{equation}
 c_{21} = a_{21}*b_{11}+a_{22}*b_{21}
\end{equation}
\begin{equation}
 c_{22} = a_{21}*b_{12}+a_{22}*b_{22}
\end{equation}
\paragraph{Question 2}
Développons les formules de Strassen :\\
\begin{align*}
c_{11} &= q_{1} - q_{3} - q_{5} + q_{7}\\
&= [a_{11}b_{22} - a_{12}b_{22}] - [a_{22}b_{11} + a_{22}b_{21}] - [- a_{11}b_{11} + a_{11}b_{22} - a_{22}b_{11} + a_{22}b_{22}] + [a_{12}b_{21} - a_{12}b_{22} + a_{22}b_{21} - a_{22}b_{22}]\\
&=  - [- a_{11}b_{11} + a_{22}b_{22}] + [a_{12}b_{21} - a_{22}b_{22}]\\
&= a_{11}b_{11} + a_{12}b_{21}
\end{align*}

\bigskip
\begin{align*}
c_{12} &= a_{11}(b_{12} + b_{22}) - (a_{11} - a_{12})b_{22}\\
&= a_{11}b_{12} + a_{11}b_{22} - a_{11}b_{22} + a_{12}b_{22}\\
&= a_{11}b_{12} + a_{12}b_{22}
\end{align*}

\bigskip
\begin{align*}
c_{21} &= (a_{21} - a_{22})b_{11} + a_{22}(b_{11} + b_{21})\\
&= a_{21}b_{11} - a_{22}b_{11} + a_{22}b_{11} + a_{22}b_{21}\\
&= a_{21}b_{11} + a_{22}b_{21}
\end{align*}

\bigskip
\begin{align*}
c_{22} &= -(a_{21} - a_{22})b_{11} - a_{11}(b_{12} + b_{22}) + (a_{11} + a_{22})(b_{22} - b_{11}) + (a_{11} + a_{21})(b_{11} + b_{12})\\
&= a_{21}b_{11} + a_{22}b_{11} - a_{11}b_{12} - a_{11}b_{12} + a_{11}b_{22} - a_{11}b_{11} + a_{22}b_{22} - a{22}b_{11} + a_{11}b_{11} + a_{11}b_{12} + a_{21}b_{11} + b_{21}b_{12}\\
&= a_{21}b_{12} + a_{22}b_{22}\\
\end{align*}
Les calculs effectués en partant des équations résultantes des formules de Strassen donnent bien le résultat de l'algorithme naïf de multiplication matricielle.

\paragraph{Question 3}
Au vue de la question 2, on peut conclure que nous ne fesons qu'une multiplication par coefficient $q$ donc un total de 7 multiplication. Cela nous fais donc économiser une multiplication. 
\end{document}          
